\documentclass[11pt,letter,]{article}
\usepackage[margin=1in]{geometry}
\newcommand*{\authorfont}{\fontfamily{phv}\selectfont}
\usepackage[]{mathpazo}
\usepackage{abstract}
\renewcommand{\abstractname}{}    % clear the title
\renewcommand{\absnamepos}{empty} % originally center
\newcommand{\blankline}{\quad\pagebreak[2]}

\providecommand{\tightlist}{%
  \setlength{\itemsep}{0pt}\setlength{\parskip}{0pt}} 
\usepackage{longtable,booktabs}

\usepackage{parskip}
\usepackage{titlesec}
\titlespacing\section{0pt}{12pt plus 4pt minus 2pt}{6pt plus 2pt minus 2pt}
\titlespacing\subsection{0pt}{12pt plus 4pt minus 2pt}{6pt plus 2pt minus 2pt}

\titleformat*{\subsubsection}{\normalsize\itshape}

\usepackage{hyperref}
\hypersetup{
    colorlinks=true,
    linkcolor=blue,
    filecolor=magenta,      
    urlcolor=blue,
    pdftitle={OPR319: Introducción a la Ciencia de Datos Aplicada con
RStudio},
    pdfpagemode=FullScreen,
}

\usepackage{titling}
\setlength{\droptitle}{-.25cm}

%\setlength{\parindent}{0pt}
%\setlength{\parskip}{6pt plus 2pt minus 1pt}
%\setlength{\emergencystretch}{3em}  % prevent overfull lines 

\usepackage[T1]{fontenc}
\usepackage[utf8]{inputenc}

\usepackage{fancyhdr}
\pagestyle{fancy}
\usepackage{lastpage}
\renewcommand{\headrulewidth}{0.3pt}
\renewcommand{\footrulewidth}{0.0pt} 
\lhead{}
\chead{}
\rhead{\footnotesize OPR319: Introducción a la Ciencia de Datos Aplicada
con RStudio -- 2024-01-01}
\lfoot{}
\cfoot{\small \thepage/\pageref*{LastPage}}
\rfoot{}

\fancypagestyle{firststyle}
{
\renewcommand{\headrulewidth}{0pt}%
   \fancyhf{}
   \fancyfoot[C]{\small \thepage/\pageref*{LastPage}}
}

%\def\labelitemi{--}
%\usepackage{enumitem}
%\setitemize[0]{leftmargin=25pt}
%\setenumerate[0]{leftmargin=25pt}




\makeatletter
\@ifpackageloaded{hyperref}{}{%
\ifxetex
  \usepackage[setpagesize=false, % page size defined by xetex
              unicode=false, % unicode breaks when used with xetex
              xetex]{hyperref}
\else
  \usepackage[unicode=true]{hyperref}
\fi
}
\@ifpackageloaded{color}{
    \PassOptionsToPackage{usenames,dvipsnames}{color}
}{%
    \usepackage[usenames,dvipsnames]{color}
}
\makeatother
\hypersetup{breaklinks=true,
            bookmarks=true,
            pdfauthor={ () and  ()},
             pdfkeywords = {},  
            pdftitle={OPR319: Introducción a la Ciencia de Datos
Aplicada con RStudio},
            colorlinks=true,
            citecolor=blue,
            urlcolor=blue,
            linkcolor=blue,
            pdfborder={0 0 0}}
\urlstyle{same}  % don't use monospace font for urls


\setcounter{secnumdepth}{0}

\usepackage{longtable}

\usepackage{graphicx}
% We will generate all images so they have a width \maxwidth. This means
% that they will get their normal width if they fit onto the page, but
% are scaled down if they would overflow the margins.
\makeatletter
\def\maxwidth{\ifdim\Gin@nat@width>\linewidth\linewidth
\else\Gin@nat@width\fi}
\makeatother
\let\Oldincludegraphics\includegraphics
\renewcommand{\includegraphics}[1]{\Oldincludegraphics[width=\maxwidth]{#1}}



\usepackage{setspace}

\title{OPR319: Introducción a la Ciencia de Datos Aplicada con RStudio}
\author{Roberto Cantillan \and Pablo Jiménez}
\date{2024-01-01}


\begin{document}  

		\maketitle
		
	
		\thispagestyle{firststyle}

%	\thispagestyle{empty}


	\noindent \begin{tabular*}{\textwidth}{ @{\extracolsep{\fill}} lr @{\extracolsep{\fill}}}


E-mail: TBD & Web: Repositorio \href{https://github.com/rcantillan/OPR319-Ciencia-de-datos-con-R}{link}\\
Office Hours: {[}Horario de oficina{]}  &  Class Hours: {[}Horario de
clases{]}\\
Office: {[}Número de oficina{]}  & Class Room: {[}Sala de clases{]}\\
	&  \\
	\hline
	\end{tabular*}
	
\vspace{2mm}
	


\hypertarget{identificaciuxf3n-de-la-actividad-curricular}{%
\section{Identificación de la Actividad
Curricular}\label{identificaciuxf3n-de-la-actividad-curricular}}

\begin{itemize}
\tightlist
\item
  \textbf{Nombre}: Ciencia de Datos Aplicada con RStudio
\item
  \textbf{Código}: OPR 319 (Equivalente a SLG-522)
\item
  \textbf{Semestre lectivo}: X
\item
  \textbf{Horas}: Presencial: 54 \textbar{} Autónomas: 96 \textbar{}
  TOTAL: 150
\item
  \textbf{Créditos SCT}: 5
\item
  \textbf{Duración}: Semestral
\item
  \textbf{Modalidad}: Presencial
\item
  \textbf{Área de Formación}: Profesional
\item
  \textbf{Requisito}: Todas las actividades curriculares aprobadas hasta
  el VIII semestre
\end{itemize}

\hypertarget{descripciuxf3n-y-caracterizaciuxf3n-de-la-actividad-curricular}{%
\section{Descripción y Caracterización de la Actividad
Curricular}\label{descripciuxf3n-y-caracterizaciuxf3n-de-la-actividad-curricular}}

La actividad curricular de Electivo III (Ciencia de datos aplicada) se
ubica en el X semestre de la carrera de Sociología y pertenece al área
de formación profesional.

En la era actual de big data, la capacidad de analizar datos y extraer
información valiosa se ha convertido en una habilidad esencial para los
científicos sociales y profesionales en diversos campos. En este
contexto, el propósito fundamental de este curso es introducir a los
estudiantes en las herramientas básicas para extraer información valiosa
a partir de datos sin procesar, a menudo generados fuera de los
tradicionales diseños de investigación científica. Esta habilidad es
crucial no solo para la investigación académica, sino también para la
toma de decisiones informadas en el sector público y privado.

El curso busca establecer una base sólida para el dominio de las
herramientas necesarias en Ciencia de Datos, utilizando R y Tidyverse,
un conjunto de software estadístico de código abierto ampliamente
reconocido en la investigación tanto académica como aplicada. Se
abordarán cada etapa del proceso de manipulación y análisis de datos,
incluyendo importación, limpieza, transformación, validación,
visualización, modelado y creación de informes automáticos y
reproducibles.

La programación en R, utilizando la interfaz de RStudio, es un
componente central de este curso. Aprender a programar no solo permite a
los estudiantes realizar análisis de datos más eficientes y
reproducibles, sino que también desarrolla habilidades de pensamiento
lógico y resolución de problemas que son valiosas en cualquier campo
profesional. Además, el dominio de R y RStudio proporciona a los
estudiantes herramientas poderosas y flexibles que son ampliamente
utilizadas tanto en la academia como en la industria.

Al finalizar el curso, los estudiantes serán capaces de:

\begin{enumerate}
\def\labelenumi{\arabic{enumi}.}
\tightlist
\item
  Analizar bases de datos de complejidad intermedia a avanzada.
\item
  Ejecutar tareas de programación para procesamiento de datos,
  relacionando niveles de estratificación con fenómenos sociales
  sustantivos.
\item
  Modelar con bases de datos simples y complejas, relacionando
  resultados con fenómenos sociales sustantivos.
\item
  Comunicar resultados de análisis mediante la creación de informes
  automatizados y reproducibles.
\end{enumerate}

Estas habilidades no solo mejorarán la capacidad de los estudiantes para
realizar investigaciones rigurosas, sino que también aumentarán
significativamente su competitividad en el mercado laboral, donde la
demanda de profesionales con habilidades en análisis de datos está en
constante crecimiento.

\hypertarget{competencias-del-perfil-de-egreso-asociadas-a-la-actividad-curricular}{%
\section{Competencias del Perfil de Egreso Asociadas a la Actividad
Curricular}\label{competencias-del-perfil-de-egreso-asociadas-a-la-actividad-curricular}}

\hypertarget{competencias-profesionales}{%
\subsection{Competencias
Profesionales}\label{competencias-profesionales}}

\begin{enumerate}
\def\labelenumi{\arabic{enumi}.}
\tightlist
\item
  Gestionar organizaciones, proyectos e intervenciones orientándose al
  trabajo colaborativo y con apertura a la diversidad social.

  \begin{itemize}
  \tightlist
  \item
    1.3. Gestionar proyectos e intervenciones sociales, promoviendo el
    trabajo en equipo interdisciplinar y colaborativo.
  \end{itemize}
\item
  Proponer iniciativas pertinentes a las demandas y necesidades de
  entidades diversas, en base a una comprensión integral de los
  fenómenos y los contextos sociales.

  \begin{itemize}
  \tightlist
  \item
    2.3. Proponer iniciativas pertinentes, relacionando críticamente
    conceptos y teorías contemporáneas de la sociología y las ciencias
    sociales.
  \end{itemize}
\end{enumerate}

\hypertarget{competencias-genuxe9ricas}{%
\subsection{Competencias Genéricas}\label{competencias-genuxe9ricas}}

\begin{enumerate}
\def\labelenumi{\arabic{enumi}.}
\setcounter{enumi}{4}
\tightlist
\item
  Realizar investigaciones que contribuyan al desarrollo del
  conocimiento científico y aplicado en contextos propios de su proceso
  formativo.

  \begin{itemize}
  \tightlist
  \item
    5.3. Desarrollar investigación aplicada, implementando los pasos del
    método científico y articulando conclusiones adecuadas y coherentes
    al proceso investigativo.
  \end{itemize}
\end{enumerate}

\hypertarget{resultados-de-aprendizaje---aprendizajes-esperados}{%
\section{Resultados de Aprendizaje - Aprendizajes
Esperados}\label{resultados-de-aprendizaje---aprendizajes-esperados}}

\begin{enumerate}
\def\labelenumi{\arabic{enumi}.}
\tightlist
\item
  Ejecutar tareas de programación para procesamiento de datos
  relacionando niveles de estratificación con fenómenos sociales
  sustantivos.
\item
  Modelar con bases de datos simples y complejas, relacionando
  resultados con fenómenos sociales sustantivos.
\item
  Comunicar resultados de análisis mediante la creación de informes
  automatizados y reproducibles.
\end{enumerate}

\hypertarget{unidades-de-aprendizaje-y-ejes-temuxe1ticos}{%
\section{Unidades de Aprendizaje y Ejes
Temáticos}\label{unidades-de-aprendizaje-y-ejes-temuxe1ticos}}

\hypertarget{semana-01-0808---1208-introducciuxf3n-a-r-y-rstudio}{%
\subsection{Semana 01, 08/08 - 12/08: Introducción a R y
RStudio}\label{semana-01-0808---1208-introducciuxf3n-a-r-y-rstudio}}

\begin{itemize}
\tightlist
\item
  Instalación de R y RStudio
\item
  Interfaz de RStudio
\item
  Conceptos básicos de R: objetos, funciones, paquetes
\item
  Tipos de datos en R: numéricos, caracteres, lógicos, factores
\item
  Operaciones aritméticas y lógicas
\item
  Vectores y matrices
\end{itemize}

\hypertarget{semana-02-1508---1908-jueves-15-feriado}{%
\subsection{Semana 02, 15/08 - 19/08: Jueves 15
feriado}\label{semana-02-1508---1908-jueves-15-feriado}}

\hypertarget{semana-03-2208---2608-operaciones-buxe1sicas-y-estructuras-de-datos}{%
\subsection{Semana 03, 22/08 - 26/08: Operaciones Básicas y Estructuras
de
Datos}\label{semana-03-2208---2608-operaciones-buxe1sicas-y-estructuras-de-datos}}

\begin{itemize}
\tightlist
\item
  Operaciones aritméticas y lógicas
\item
  Vectores y matrices
\item
  Listas y data frames
\item
  Indexación y subconjuntos
\end{itemize}

\hypertarget{semana-04-2908---0209-importaciuxf3n-y-exportaciuxf3n-de-datos}{%
\subsection{Semana 04, 29/08 - 02/09: Importación y Exportación de
Datos}\label{semana-04-2908---0209-importaciuxf3n-y-exportaciuxf3n-de-datos}}

\begin{itemize}
\tightlist
\item
  Lectura de archivos CSV, Excel, y otros formatos
\item
  Escritura de datos en diferentes formatos
\item
  Conexión con bases de datos
\end{itemize}

\hypertarget{semana-05-0509---0909-introducciuxf3n-a-tidyverse}{%
\subsection{\texorpdfstring{Semana 05, 05/09 - 09/09: Introducción a
\texttt{Tidyverse}}{Semana 05, 05/09 - 09/09: Introducción a Tidyverse}}\label{semana-05-0509---0909-introducciuxf3n-a-tidyverse}}

\begin{itemize}
\tightlist
\item
  Filosofía de \texttt{Tidyverse}
\item
  Pipes (\%\textgreater\%) y su uso
\item
  Introducción a \texttt{dplyr}: \texttt{select()}, \texttt{filter()},
  \texttt{mutate()}
\end{itemize}

\hypertarget{semana-06-1209---1609-manipulaciuxf3n-de-datos-con-dplyr-i}{%
\subsection{\texorpdfstring{Semana 06, 12/09 - 16/09: Manipulación de
Datos con \texttt{dplyr}
I}{Semana 06, 12/09 - 16/09: Manipulación de Datos con dplyr I}}\label{semana-06-1209---1609-manipulaciuxf3n-de-datos-con-dplyr-i}}

\begin{itemize}
\tightlist
\item
  Agrupación y resumen de datos: \texttt{group\_by()} y
  \texttt{summarize()}
\item
  Ordenamiento de datos: \texttt{arrange()}
\item
  Creación de nuevas variables (avanzado): \texttt{mutate()},
  \texttt{case\_when()}
\end{itemize}

\hypertarget{semana-07-1909---2309-feriado-fiestas-patrias}{%
\subsection{Semana 07, 19/09 - 23/09: Feriado Fiestas
Patrias}\label{semana-07-1909---2309-feriado-fiestas-patrias}}

\hypertarget{semana-08-2609---3009-manipulaciuxf3n-de-datos-con-dplyr-ii}{%
\subsection{\texorpdfstring{Semana 08, 26/09 - 30/09: Manipulación de
Datos con \texttt{dplyr}
II}{Semana 08, 26/09 - 30/09: Manipulación de Datos con dplyr II}}\label{semana-08-2609---3009-manipulaciuxf3n-de-datos-con-dplyr-ii}}

\begin{itemize}
\tightlist
\item
  Joins: \texttt{inner\_join()}, \texttt{left\_join()}, etc.
\item
  Operaciones de conjunto: \texttt{union()}, \texttt{intersect()},
  \texttt{setdiff()}
\item
  Manejo de datos faltantes
\end{itemize}

\hypertarget{semana-09-0310---0710-transformaciuxf3n-de-datos-con-tidyr}{%
\subsection{\texorpdfstring{Semana 09, 03/10 - 07/10: Transformación de
Datos con
\texttt{tidyr}}{Semana 09, 03/10 - 07/10: Transformación de Datos con tidyr}}\label{semana-09-0310---0710-transformaciuxf3n-de-datos-con-tidyr}}

\begin{itemize}
\tightlist
\item
  Datos tidy y su importancia
\item
  Funciones \texttt{pivot\_longer()} y \texttt{pivot\_wider()}
\item
  Separación y unión de columnas: \texttt{separate()} y \texttt{unite()}
\end{itemize}

\hypertarget{semana-10-1010---1410-iteraciuxf3n-y-automatizaciuxf3n-con-purrr}{%
\subsection{\texorpdfstring{Semana 10, 10/10 - 14/10: Iteración y
Automatización con
\texttt{purrr}}{Semana 10, 10/10 - 14/10: Iteración y Automatización con purrr}}\label{semana-10-1010---1410-iteraciuxf3n-y-automatizaciuxf3n-con-purrr}}

\begin{itemize}
\tightlist
\item
  Conceptos de programación funcional
\item
  Familia de funciones \texttt{map()}
\item
  Uso de \texttt{purrr} con \texttt{dplyr}
\end{itemize}

\hypertarget{semana-11-1710---2110-visualizaciuxf3n-de-datos-i-fundamentos-de-ggplot2}{%
\subsection{\texorpdfstring{Semana 11, 17/10 - 21/10: Visualización de
Datos I: Fundamentos de
\texttt{ggplot2}}{Semana 11, 17/10 - 21/10: Visualización de Datos I: Fundamentos de ggplot2}}\label{semana-11-1710---2110-visualizaciuxf3n-de-datos-i-fundamentos-de-ggplot2}}

\begin{itemize}
\tightlist
\item
  Gramática de gráficos
\item
  Capas en \texttt{ggplot2}
\item
  Tipos básicos de gráficos: dispersión, líneas, barras
\end{itemize}

\hypertarget{semana-12-2410---2810-visualizaciuxf3n-de-datos-ii-ggplot2-avanzado}{%
\subsection{\texorpdfstring{Semana 12, 24/10 - 28/10: Visualización de
Datos II: \texttt{ggplot2}
Avanzado}{Semana 12, 24/10 - 28/10: Visualización de Datos II: ggplot2 Avanzado}}\label{semana-12-2410---2810-visualizaciuxf3n-de-datos-ii-ggplot2-avanzado}}

\begin{itemize}
\tightlist
\item
  Faceting
\item
  Temas y personalización de gráficos
\item
  Combinación de múltiples gráficos
\end{itemize}

\hypertarget{semana-13-3110---0411-feriado}{%
\subsection{Semana 13, 31/10 - 04/11:
Feriado}\label{semana-13-3110---0411-feriado}}

\hypertarget{semana-14-0711---1111-introducciuxf3n-al-modelado-estaduxedstico}{%
\subsection{Semana 14, 07/11 - 11/11: Introducción al Modelado
Estadístico}\label{semana-14-0711---1111-introducciuxf3n-al-modelado-estaduxedstico}}

\begin{itemize}
\tightlist
\item
  Regresión lineal con \texttt{lm()}
\item
  Interpretación de resultados
\item
  Diagnósticos básicos de modelos
\end{itemize}

\hypertarget{semana-15-1411---1811-modelos-avanzados-con-tidymodels}{%
\subsection{\texorpdfstring{Semana 15, 14/11 - 18/11: Modelos Avanzados
con
\texttt{tidymodels}}{Semana 15, 14/11 - 18/11: Modelos Avanzados con tidymodels}}\label{semana-15-1411---1811-modelos-avanzados-con-tidymodels}}

\begin{itemize}
\tightlist
\item
  Introducción a \texttt{tidymodels}
\item
  Preprocesamiento de datos
\item
  Entrenamiento y evaluación de modelos
\end{itemize}

\hypertarget{semana-16-2111---2511-reportes-automatizados-i-introducciuxf3n-a-rmarkdown}{%
\subsection{\texorpdfstring{Semana 16, 21/11 - 25/11: Reportes
Automatizados I: Introducción a
\texttt{RMarkdown}}{Semana 16, 21/11 - 25/11: Reportes Automatizados I: Introducción a RMarkdown}}\label{semana-16-2111---2511-reportes-automatizados-i-introducciuxf3n-a-rmarkdown}}

\begin{itemize}
\tightlist
\item
  Sintaxis básica de \texttt{Markdown}
\item
  Chunks de \texttt{R} en \texttt{RMarkdown}
\item
  Generación de reportes en diferentes formatos (PDF, HTML)
\end{itemize}

\hypertarget{semana-17-2811---0212-reportes-automatizados-ii-quarto}{%
\subsection{\texorpdfstring{Semana 17, 28/11 - 02/12: Reportes
Automatizados II:
\texttt{Quarto}}{Semana 17, 28/11 - 02/12: Reportes Automatizados II: Quarto}}\label{semana-17-2811---0212-reportes-automatizados-ii-quarto}}

\begin{itemize}
\tightlist
\item
  Parámetros en \texttt{Quarto}
\item
  Creación de presentaciones con \texttt{Quarto}
\item
  Introducción a \texttt{Quarto}
\end{itemize}

\hypertarget{semana-18-0512---0912-proyecto-final---trabajo-en-clase}{%
\subsection{Semana 18, 05/12 - 09/12: Proyecto Final - Trabajo en
Clase}\label{semana-18-0512---0912-proyecto-final---trabajo-en-clase}}

\begin{itemize}
\tightlist
\item
  Sesión de trabajo guiado para el proyecto final
\item
  Consultas y asesorías individuales
\end{itemize}

\hypertarget{semana-19-1212---1612-proyecto-final---presentaciones}{%
\subsection{Semana 19, 12/12 - 16/12: Proyecto Final -
Presentaciones}\label{semana-19-1212---1612-proyecto-final---presentaciones}}

\begin{itemize}
\tightlist
\item
  Presentación de proyectos finales por parte de los estudiantes
\item
  Retroalimentación y discusión
\end{itemize}

\hypertarget{estrategias-de-enseuxf1anza-y-aprendizaje}{%
\section{Estrategias de Enseñanza y
Aprendizaje}\label{estrategias-de-enseuxf1anza-y-aprendizaje}}

\begin{itemize}
\tightlist
\item
  Clases expositivas dialogadas
\item
  Aprendizaje Basado en Problemas
\item
  Aprendizaje Colaborativo o Cooperativo
\item
  Talleres
\item
  Trabajo en laboratorio de computación
\end{itemize}

\hypertarget{procedimientos-de-evaluaciuxf3n-de-aprendizajes}{%
\section{Procedimientos de Evaluación de
Aprendizajes}\label{procedimientos-de-evaluaciuxf3n-de-aprendizajes}}

\begin{enumerate}
\def\labelenumi{\arabic{enumi}.}
\tightlist
\item
  Tarea de programación I (35\%)

  \begin{itemize}
  \tightlist
  \item
    Instrumento: Rúbrica e Informe (Escala de apreciación)
  \item
    Contenidos: Importación y exportación, agrupación y resumen de datos
    y manipulación de datos de datos con \texttt{dplyr}, manipulación de
    datos II (unión), transformación de datos (\texttt{tidyr}), manejo
    de datos faltantes.
  \end{itemize}
\item
  Tarea de programación II (35\%)

  \begin{itemize}
  \tightlist
  \item
    Instrumento: Rúbrica y Diagrama de flujo (Pauta de cotejo)
  \item
    Contenidos: Iteración y automatización (\texttt{purrr}), creación de
    gráficas (\texttt{ggplot}).
  \end{itemize}
\item
  Tarea de programación III (30\%)

  \begin{itemize}
  \tightlist
  \item
    Instrumento: Rúbrica y Autoevaluación de proceso (Pauta de cotejo)
  \item
    Contenidos: Modelado estadístico con \texttt{tidymodels}
    (procesamiento y uso de resultados), reportes con \texttt{Quarto}.
  \end{itemize}
\end{enumerate}

\hypertarget{recursos-de-infraestructura}{%
\section{Recursos de
Infraestructura}\label{recursos-de-infraestructura}}

\begin{itemize}
\tightlist
\item
  Sala de clases con proyector y audio
\item
  Sala de clases con computadores con R y RStudio instalado
\item
  Acceso a UCM Virtual (Plataforma Web LMS)
\end{itemize}

\hypertarget{recursos-bibliogruxe1ficos}{%
\section{Recursos Bibliográficos}\label{recursos-bibliogruxe1ficos}}

\hypertarget{buxe1sica-obligatoria}{%
\subsection{Básica Obligatoria}\label{buxe1sica-obligatoria}}

\begin{itemize}
\tightlist
\item
  Wickham, H., Çetinkaya-Rundel, M., \& Grolemund, G. (2023). R for data
  science. O'Reilly Media, Inc.~Versión en español: R para ciencia de
  datos: https://es.r4ds.hadley.nz/
\item
  Ismay, C., \& Kim, A. Y.-S. (2020). Statistical inference via data
  science: A ModernDive into R and the Tidyverse. CRC Press / Taylor \&
  Francis Group. (versión gitbook de libre acceso:
  https://moderndive.com/)
\end{itemize}

\hypertarget{complementaria}{%
\subsection{Complementaria}\label{complementaria}}

\begin{itemize}
\tightlist
\item
  Imai, K., \& Williams, N. W. (2022). Quantitative Social Science: An
  Introduction in Tidyverse. Princeton University Press.
\item
  Wickham, H. (2019). Advanced r. CRC press. (link a versión on line de
  libre acceso https://adv-r.hadley.nz/)
\end{itemize}

\hypertarget{otros-recursos}{%
\section{Otros Recursos}\label{otros-recursos}}

\begin{itemize}
\tightlist
\item
  UCM Virtual (Plataforma Web LMS)
\item
  Artículos científicos (Digital)
\item
  Repositorio SciELO (Página Web)
\item
  SIBIB (Página Web)
\item
  Repositorio GitHub (Página Web)
\item
  Base de datos de acceso público (Página Web de distintas
  organizaciones)
\end{itemize}




\end{document}

\makeatletter
\def\@maketitle{%
  \newpage
%  \null
%  \vskip 2em%
%  \begin{center}%
  \let \footnote \thanks
    {\fontsize{18}{20}\selectfont\raggedright  \setlength{\parindent}{0pt} \@title \par}%
}
%\fi
\makeatother
